%%%%%%%%%%%%%%%%%%%%%%%%%%%%%%%%%%%%%%%%%%%%%%%%%%%%%%%%%%%%%%%%%%%%%%%%%%%%%%%
% Chapter 2: Fundamentos Teóricos 
%%%%%%%%%%%%%%%%%%%%%%%%%%%%%%%%%%%%%%%%%%%%%%%%%%%%%%%%%%%%%%%%%%%%%%%%%%%%%%%

%++++++++++++++++++++++++++++++++++++++++++++++++++++++++++++++++++++++++++++++

En este capítulo se han de presentar los antecedentes teóricos y prácticos que
apoyan el tema objeto de la investigación.

%++++++++++++++++++++++++++++++++++++++++++++++++++++++++++++++++++++++++++++++

\section{Primer apartado del segundo capítulo}
\label{2:sec:1}
  Primer párrafo de la primera sección.

\section{Segundo apartado del segundo capítulo}
\label{2:sec:2}
  Primer párrafo de la segunda sección.

En \LaTeX{}~\cite{Lam:86} es sencillo escribir expresiones
matem\'aticas como $a=\sum_{i=1}^{10} {x_i}^{3}$
y deben ser escritas entre dos s\'imbolos \$-
Los super\'indices se obtienen con el s\'imbolo \^{}, y
los sub\'indices con el s\'imbolo \_.
Por ejemplo: $x² \times y^{\alpha + \beta}$.
Tambi\'en se pueden escribir f\'ormulas centradas:
\[h^2=a^2 + b^2 \]
